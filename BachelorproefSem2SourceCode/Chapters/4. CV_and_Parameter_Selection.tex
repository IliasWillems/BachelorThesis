\section{Cross-Validation and parameter selection} \label{sec:4.CV&PS}

Cross-Validation can be used as a method of assessing the quality of a model, also known as \textit{Model assessment}. It can also be used to select the appropriate level of flexibility, also known as \textit{Model selection}. We will use Cross-Validation to select the tuning parameter $\lambda$ which minimizes the MSE. \\

In this section we will first show some methods of applying cross-validation and parameter tuning. Later, we will apply cross-validation to our models of the different types of regression we used in section~\ref{sec:3.PR}. We will then compare these types of regression in section~\ref{sec:4.CV&PS}. 

%--------------------------------------------
% --- Different types of Cross-Validation ---
%--------------------------------------------

\subsection{Different types of Cross-Validation}

% - K-fold Cross-Validation - 

\subsubsection{K-fold Cross-Validation}

% - Leave-one-out Cross-Validation (LOOCV) - 

\subsubsection{Leave-one-out Cross-Validation (LOOCV)}

%--------------------------------------------
% --- Applying 10-fold Cross-Validation ---
%--------------------------------------------

\subsection{Applying 10-fold Cross-Validation}

% - Linear regression - 

\subsubsection{Linear regression}

% - Ridge regression - 

\subsubsection{Ridge regression}

% - The Lasso - 

\subsubsection{The Lasso}